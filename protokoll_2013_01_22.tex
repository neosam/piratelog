
\documentclass{protokoll}

\renewcommand{\Sitzungstermin}{22.01.2013}

\begin{document}

\Überschrift

\section{Teilnehmer}

\begin{Teilnehmer}{Vorstand}
        \person Christian "BimBimBimBim ;-)" Reidel
        \person Heide-Marie Weiherer
        \person Otto Pittner verhindert!
        \person Walter Weber
\end{Teilnehmer}

\begin{Teilnehmer}{Beauftragte}
        \person Captain\_Jack\_S
        \person Neosam
\end{Teilnehmer}

\Sitzungsleitung{Christian "Bim" Reidel}
\Zeit{19:30}{21:15}

\section{Tagesordnung}

Hinweis auf die Aufzeichnung

\subsection{TOP 1: Tätigkeitsberichte aus dem Vorstand}

\subsubsection{1. Vorstand (Bim)}

\Tätigkeit{16./17.1.}{Mitläufersuche}
\Tätigkeit{18.1.}{Banktermin}
\Tätigkeit{}{Orga f. evtl. Bezirksgeschäftsstelle}
\Tätigkeit{20.01.}{PK in DEG wg. Bündnis gg. Studiengebühren}
\Tätigkeit{20.01.}{AV REG/Wahlparty}
\Tätigkeit{21.01.}{Vorstandstreffen in SR}

\subsubsection{2. Vorstand (Flusspiratin)}

\Tätigkeit{18.12.12}{Besuch Stammtisch Passau}
\Tätigkeit{19.12.12}{Stammtisch Regen/FRG/Deggendorf in Regen}
\Tätigkeit{20.12.12}{Stammtisch Landshut}
\Tätigkeit{27.12.12}{Stammtisch Straubing}
\Tätigkeit{04.01.13}{Besuch "Nautic" Straubing, Jahresrückblick}
\Tätigkeit{09.01.13}{Stammtisch Regen/FRG/Deggendorf in Deggendorf}
\Tätigkeit{15.01.13}{Telko BzV - entschuldigt}
\Tätigkeit{17.01.13}{Stammtisch Landshut}
\Tätigkeit{20.01.13}{Gründung KV-Bayerwaldpiraten, Wahlleitung; AV Bundestag Straubing - Fotos für Presse}
\Tätigkeit{21.01.13}{Vorstandstreffen Orga zur Wahl in Straubing}
\Tätigkeit{26.01.}{Teilnahme am Infostand in Passau}
\Tätigkeit{}{Kontakt hergestellt zu unser Radio}
 
\subsubsection{Schatzie (Walter)}
\Tätigkeit{}{mTAN von der Sparkasse erhalten, noch keinen SAGE Zugang, daher keine Buchung möglich, bleibe aber dran.}
\Tätigkeit{15.01.13}{Pressekonferenz des Bündnisses gegen Volksbegehren in Straubing}
\Tätigkeit{20.01.13}{KV-Gründung in Regen, AV -BTW in Regen}
\Tätigkeit{21.01.13}{Orgatreffen}


\subsubsection{GenSek (Otto)}
\textbf{für heute entschuldigt}

\Tätigkeit{17.01.13}{Besuch Stammtisch Landshut}
\Tätigkeit{18.01.13}{Vorbereitung auf KV Gründung und AV Bdt am Sonntag, Abklärung offener Fragen mit Spinni, Vorbereitung von Formularen, Satzungen, TO´s, GO´s,usw.}
\Tätigkeit{20.01.13}{Akkreditierung für KV Gründung Bayerwaldpiraten, Akkreditierung für AV zur Btw-Direktkanditaten Straubing/Regen, Organisation und Anfertigung sowie Zusammenstellung aller nötigen Formulare zum weiterleiten an die zuständigen Stellen.}

\subsubsection{Fragen an den Vorstand}

\begin{itemize}
        \item Welche Unterschriftten fehlen fuer den Bankzugang
        \begin{itemize}
          \item Bim: Die erstmals eingereichten Unterlagen sind bei der Bank verloren gegangen.  Die Unterlagen sind unterwegs.
          \item Walter muss noch unterschreiben, wird das Formular zurückschicken und schließlich von der Bank die Karte erhalten.
        \end{itemize}
        \item Ist eine Kostenübernahme für einen Pavillon für die HSG-Piraten denkbar, wenn der Betrag entsprechend gespendet wird? (evtl. Verzicht auf Erstattung von Aufwendungen)
        \begin{itemize}
          \item Wird später weiter besprochen
        \end{itemize}
\end{itemize}

\subsection{TOP 2: Infos von den Stammtischen }

\subsubsection{Stammtisch Landshut}
\begin{itemize}
\item Infostand Aktionsbündnis gegen Studiengebühren. Sehr gut gelaufen.
\item Besuch von Otto und Heidi am Stammtisch.
\item FK Thema/ Bims Frage bzgl. der Vereinbarkeit der Ämter nochmal behandelt und das Protokoll wurde korrigiert
\item KV Gründung wird auf den 28.02.2013 terminiert.
\item AV wird nach der Gründung diskutiert
\item Bim: Warum keine KV Gründung und AV am Wochenende?
  \begin{itemize}
  \item Weil kein Wochenende dafür geopfert werden soll und ein Donnerstag vorgezogen wird.
  \end{itemize}
\end{itemize}

\subsubsection{Stammtisch Straubing}
\begin{itemize}
\item Teilnahme am Bündnistreffen gegen Studiengebühren am 15.1.
\item AV Direktkandidaten Landtag/Bezirk am 24.1. 19.00 Uhr Landshuter Hof
\item Infostand am 26.1. im Rahmen des Straubinger Bündnisses gegen Studiengebühren
\item in Planung: ÖPNV Kostenanalyse als Basis zur Prüfung eines fahrscheinlosen ÖPNV für Straubing, 
\item Forderung des Verzichts der Videoüberwachung während des Gäubodenvolksfestes 2013 wegen mangelnder Erklärung des Polizeipräsidiums Niederbayern/Oberpfalz
\end{itemize}

\subsubsection{Stammtisch Deggendorf}
\begin{itemize}
\item KV Gründung Bayerwaldpiraten
\end{itemize}

\subsubsection{Stammtisch Passau}
\begin{itemize}
\item Begrüßung und Einführung eines neuen Interessenten
\item KV Gründung am 5.3.13
\item Infostand kommenden Samstag in Planung
  \begin{itemize}
  \item Wir haben Infomaterial gegen die Studiengebühren, Flaggen, Beachflag, Biertisch
  \item Infostand ist angemeldet, Rechnung bezahlt HSG
  \item Bitte eintragen auf: \url{http://doodle.com/mpigr5h5h6fwveur}
  \item Wir brauchen noch einen (Bier-)tisch, evtl Jiri, falls man ihn erwischt
  \end{itemize}
\end{itemize}

\subsubsection{Stammtisch Bayerwaldpiraten (REG/FRG)}
\begin{itemize}
\item KV Gründung Bayerwaldpiraten
\end{itemize}

\subsubsection{Stammtisch Rottal (im Aufbau)}
Heute Stammtisch, Rainer ist vor Ort

\subsubsection{Stammtisch Kelheim}
Wegen dem Zwischendurch der Vorstandssitzung hat seit der letzten Sitzung kein Stammtisch stattgefunden.

\subsubsection{Stammtisch Vilsbiburg}
Findet zur Zeit nicht statt.

\subsection{Beauftragte}

\subsubsection{Homepage}
\begin{itemize}
\item Anlegen von drei Autoren auf der Niederbayernhomepage am 17.01.2013 (niederbayern,hp)
\item Peda zum Redakteur der Homepage Niederbayern befördert am 20.01.2013 (niederbayern,hp)
\item Automatisches Entscheidungsbuch für den Vorstand von Niederbayern erstellt (bzw vom LV Bayern geklaut hihihi) am 22.01.2013 (niederbayern,wiki)
\item Meinungsbild ob Generierung der Protokoll PDFs mittels OpenOffice vorerst ok ist, da eine \LaTeX-Vorlage scheinbar nicht existiert und erst erstellt werden muss.
  \begin{itemize}
  \item Peda und Matthias haben schon eine Vorlage gebastelt, der Punkt fällt also weg
  \end{itemize}
  \item Ich muss noch den Hinweis zur Aufzeichnung ins Wiki und HP einpflegen
\end{itemize}


\subsection{TOP 3: Orga AVs}
Orga AVs: Der Vorstand und Heiko hat sich gestern in Straubing zur Vorbereitung
getroffen.  

\textbf{Ergebnisse:}

bevorstehende fixe Aufstellungsversammlungen sind \\
in Straubing am 24.01.2013 im Gasthaus Landshuter Hof um 19.00 Uhr, Akkreditierung um 18.00 Uhr.\\
in Deggendorf am Sonntag, 27.01.2013 Uhrzeit: 11:30 Uhr. 
Veranstaltungsort: Orangenzimmer Östlicher Stadtgraben 13, 94469 Deggendorf.\\
In Regen wird Otto binnen 2 Wochen einen Termin machen.

Offene Aufstellungsversammlungen sind noch:\\
Kehlheim, dafür haben wir den Termin 10.03.2013 festgelegt Lokal wird noch gesucht in Zusammenarbeit mit dem Kehlheimer Stammtisch.

Landshut, sollte von den Landshutern Piraten selbst zeitnah terminiert werden

In Rottal-Inn ist gerade eben der Stammtisch und Guido und Rainer besprechen das Thema heute.

DGF / Vilsbiburg ist von uns am 17.03.2013 terminiert, das Lokal wird noch eruiert.

\textbf{Unterschriften:}

Je Wahl sind in unserem Bezirk 923 Unterstützer - Unterschriften notwendig,
Die Unterstützer – Unterschriften können wir erst nach abgeschlossenen AV´s machen, denn somit stehen die Kandidaten fest.

Unterstützerunterschriften müssen bis spätestens 72 Tage vor der Wahl beim Wahlleiter vorliegen

Bim und Heidi werden die Piraten in den weißen Flecken persönlich anrufen.

Frage:  Ist das rechtlich in Ordnung, die Piraten anzurufen?\\
Bim: Ja, ist in Ordnung.  Sie werden als Parteimitglieder angerufen.  Wir halten dies für die beste Lösung, 
     die Piraten zu erreichen.\\
Ahoi will rechtliche Absicherung und bevorzugt Alternativen.(konktrete bessere Vorschläge gerne!)\\
Heidi:  Ist vergleichbar mit einer Vereinsmitgliederliste. (nicht jeder hat Email)\\
Tobias: ist ein Pirat ohne Email ein Pirat?\\
Ahoi: bring nochmals sine bedenken zum Ausdruck :-)\\
Bim merkt an, dass es die aufwendigste Lösung ist.\\
Heidi: man merkt dann auch wie in Zukuft der kontakt sein soll.\\

\subsection{TOP 4} 
Einreichung von Anträgen nur über Pad?
Entsprechende Hinweise im Wiki und auf der Vorstandsseite mit Link zum aktuellen Protokollpad

\textbf{Nach hinten verschoben}

\subsection{TOP 5} 

Struktur der Entscheidungen: Der Vorstand möge beschließen, in den
Vorstandstelkos nur Anträge zu beschließen, welche fristgemäß 24h
vorher eingereicht wurden. Eine Änderung des Antrags im Schrift und
Wortlaut ist während der Vorstandstelko nicht möglich. Dies bedeutet,
dass Anträge in dem Schrift und Wortlaut und in dem Verständnis des
Einzelnen entschieden werden in dem diese fristgerecht
vorliegen. Sollte der Steller des Antrags mit dem Ergebnis unzufrieden
sein, ist eine Diskussion möglich, muß aber durch die einfache
Mehrheit jeweils beschlossen werden. Wird ein abgelehnter Antrag
erneut in geänderter Form eingereicht, sollte eine Versionsänderung
vom Antragssteller dokumentiert und nachvollziehbar zur Verfügung
gestellt werden.

Begründung:
Es ist für eine Bewertung und Umsetzung der entschiedenen Anträge
besser, in diesem Modus zu arbeiten, die Qualität der eingereichten
Anträge wird zunehmen.

\textbf{zurückgezogen}


\subsection{TOP 6} 
\begin{itemize}
\item Wanninger enthält Kündigung an den BzV.  Walter will klären, wie damit umzugehen ist
  \begin{itemize}
  \item Eine Schulung wird stattfinden, die das regelt
  \end{itemize}
\end{itemize}

\subsection{TOP 7}

Bim erläutert: Zwischenbericht evtl. Bezirksgeschäftsstelle\\
Wir können die nächsten 1½ Jahre eine mietfreie Bezirksgeschäftsstelle in Passau erhalten.  
Als Untermiete eines Ladens.

\subsection{TOP 8 Anträge}

\begin{Antrag}
        {Beschleunigung der Listenaufstellung zur Landtags- und Bezirkstagswahl 2013}
        {\zurückgezogen}

\Antragssteller{fLOh}

\begin{Text}
Möglicherweise wird es in Bayern zu vorgezogenen Landtagswahlen
kommen. Aus diesem Grunde ist es angezeigt, dass so schnell als
möglich die Direktkandidaten zur Landtags-/Bezirkstagswahl sowie die
jeweiligen Listen aufgestellt werden.

Der Bezirksvorstand möge aktiv auf mögliche Direkt-/Listenkandidaten
zugehen und die Organisation der jeweiligen Aufstellungsversammlungen
übernehmen bzw. koordinieren. Sollten keine Kandidaten für einen
Stimmkreis gefunden werden können, soll die Möglichkeit
stimmkreisfremder Bewerber geprüft werden. Falls auch in dieser Weise
kein Direktkandidat gefunden werden kann, ist dies akzeptabel und es
soll mit der Planung der Listenaufstellungsversammlungen fortgefahren
werden.

Hierbei sind folgende Fristen zu beachten:
\begin{itemize}
\item Direktkandidaten (LTW/BzTW): bis spätestens 24.02.2013
\item Listenaufstellungsversammlung (LTW/BzTW): bis spätestens 10.03.2013
\end{itemize}

Es sind die Ladungsfristen zu beachten.

Für die Listenaufstellungsversammlungen, insbesondere für die
Landtagswahl, ist frühzeitig ein rechtsicherer und praktikabler
Wahlmodus zu bestimmen und in angemessener Frist vorab bekannt zu
geben, sodass eventuelle Einwände oder Verbesserungsvorschläge noch
miteinfließen können.

Nach Aufstellung der Liste für die Landtags-/Bezirkstagswahl soll
umgehend in Zusammenarbeit mit den Stammtischen bzw. Kreisverbänden
die Unterstützerschriftensammlung organisiert werden. Die genaue
Erklärung des Ablaufs obliegt hierbei dem Bezirksvorstand. Er dient
auch als verbindlicher Ansprechpartner bei Nachfragen sowohl
organisatorischer als auch rechtlicher Natur.

Im Anschluß an die Listenaufstellung kümmert sich der Bezirksvorstand,
zeitgleich zur Unterstützerschriftensammlung, um die Planung,
Organisation und Anbahnung der nötigen Presse-/Öffentlichkeitsarbeit
für die gewählten Direktkandidaten (LTW/BzTW), sowie die beiden
Spitzenkandidaten (Liste LTW/BzTW). Diesbezüglich soll mit den
Stammtischen bzw. Kreisverbänden zusammengearbeitet werden.
\end{Text}

\begin{Begründung}
Medienberichten zufolge besteht die Möglichkeit vorgezogener Neuwahlen
des Landtages in Bayern. Daher ist die Beschleunigung der ausstehenden
Aufstellungsversammlungen und Wahlkampfvorbereitungen angezeigt.  Das
Risiko eines Scheiterns des Wahlkampfes ist zu minimieren, da ein
negativer Verlauf des Wahlkampfes und der Wahlen den Erfolg der
Piraten bayernweit gefährden kann.  Dieser Antrag soll die Chancen der
niederbayerischen und bayerischen Piraten hinsichtlich eines
wünschenswerten Verlaufs der Wahlen maximieren.
\end{Begründung}

\Abstimmung{Christian ''Bim'' Reidel}{\neutral}
\Abstimmung{Heide-Marie Weiherer}{\neutral}
\Abstimmung{Walter Weber}{\neutral}
\Abstimmung{Otto Pittner}{\neutral}

\Zusatzinfo{zurückgezogen}

\end{Antrag}


\begin{Antrag}
        {Alternierende Sitzungsleitung}
        {\abgelehnt}

\Antragssteller{MatthiasZ}

\begin{Text}
Der Vorstand möge beschließen, die Sitzungsleitung der Vorstandssitzung alternierend vorzunehmen.
\end{Text}

\begin{Begründung}
Da sich in der Vergangenheit herausstellte, dass der bisherige
Sitzungsleiter weniger sachliche, sondern eher emotionale und
personenbezogene Verhaltensweisen zeigte, erscheint eine alternierende
Sitzungsleitung angebracht.
\end{Begründung}

\Abstimmung{Christian "Bim" Reidel}{\neutral}
\Abstimmung{Heide-Marie Weiherer}{\nein}
\Abstimmung{Walter Weber}{\nein}
\Abstimmung{Otto Pittner}{\nichtda}

\end{Antrag}


\begin{Antrag}
        {Piraten-Grundausstattung für jeden KV}
        {\abgelehnt}

\Antragssteller{F6F}

\begin{Text}
Der Vorstand möge beschließen jedem KV zu seiner Gründung eine
minimale Piraten-Ausstattung zu schenken, bestehend aus einer Fahne,
Grundsatzprogrammen und einem Tischwimpel (sofern noch nicht
vorhanden).
\end{Text}

\begin{Begründung}
\begin{itemize}
\item Wir sind Piraten und brauchen piratiges Zeug
\item Fahnen und Wimpel sind nachhaltige Parteiwerbung, da sie immer wieder verwendet werden können.
\item der Wunsch nach Nachhaltigkeit ist in der bayrischen Piratenpartei tief verwurzelt
\end{itemize}
\end{Begründung}

\Abstimmung{Christian "Bim" Reidel}{\nein}
\Abstimmung{Heide-Marie Weiherer}{\nein}
\Abstimmung{Walter Weber}{\nein}
\Abstimmung{Otto Pittner}{\nichtda}

\end{Antrag}



\begin{Antrag}
        {1500 Grundsatzprogramme für den Bezirk}
        {\angenommen}

\Antragssteller[22.1.2013]{F6F}
\Unterstützer{Berberic, Peda}

\begin{Text}
Der Vorstand möge für den Bezirksverband und alle seine Untergliederungen 1500 Grundsatzprogramme kaufen.
\end{Text}

\begin{Begründung}
\begin{itemize}
\item Mitglieder der PP müssen sich und ihre Bekannten und Freunde informieren können.
\item An Infoständen ist das Grundsatzprogramm A und O. 
\item Kein Mao ohne Bibel.
\item Die Programme können in kleinen packen an Stammtsiche und KVs verteilt werden.
\item Die Programme laufen nicht ab und können auch noch verteilt werden wenn sie nicht mehr brandneu sind!
\item Ergänzung: Kosten ca 300 Euro, Grundsatzprogramme im Zeitungsformat
\end{itemize}
\end{Begründung}

\Abstimmung{Christian "Bim" Reidel}{\ja}
\Abstimmung{Heide-Marie Weiherer}{\ja}
\Abstimmung{Walter Weber}{\ja}
\Abstimmung{Otto Pittner}{\nichtda}

\end{Antrag}


\begin{Antrag}
        {Regelmäßige Schatzmeisterberichte in den Vorstandssitzungen}
        {\angenommen}

\Antragssteller[22.1.2013]{M G Berberich}
\Unterstützer{F6F, Peda, Ahoi}

\begin{Text}
Der Vorstand möge beschließen, daß der Schatzmeister auf den
Vorstandssitzungen einen kurzen Überblick über Vorgänge in seinem
Zuständigkeitsbereich gibt.
\end{Text}

\begin{Begründung}
Im Zuge des Transparenzgedankens sollten auch Finanzangelegenheiten
offengelegt werden, soweit es Datenschutzaspekte zulassen.

Deswegen sollte der Schatzmeister auf den Vorstandssitzungen eine
''Zusammenfassung'' seiner Tätigkeiten, Einnahmen und Ausgaben
abgeben. Dabei sollte er sich Bemühen durch Gruppierung der Vorgänge
einen ''Überblick'' darüber zu geben welche Einnahmen der
Bezirksverband hat und wofür Mittel ausgegeben wurden ohne
schutzwürdige personenbezogene Daten preiszugeben.

Ziel soll es u. A. sein, der Basis die Möglichkeit zu geben, sich
davon zu überzeugen, daß der Vorstand mit den Mitgliedsbeiträgen und
Spenden verantwortungsvoll umgeht und sie für die Ziele der Partei
einsetzt.
\end{Begründung}

\Abstimmung{Christian "Bim" Reidel}{\ja}
\Abstimmung{Heide-Marie Weiherer}{\ja}
\Abstimmung{Walter Weber}{\ja}
\Abstimmung{Otto Pittner}{\nichtda}

\end{Antrag}


\begin{Antrag}
        {Einreichung von Anträgen nur über Wiki}
        {\angenommen}

\Antragssteller{M G Berberich}
\Unterstützer{Peda, Ahoi}

\begin{Text}
Anträge sollten nur über das Wiki, nicht über das Pad gestellt werden.
\end{Text}

\begin{Begründung}
Die Tatsache, daß Anträge über Wiki und über Pad gestellt werden, kann
– wie bei der letzten Vorstandssitzung fast passiert – dazu führen,
daß Anträge übersehen werden, zudem entsteht Aufwand die Anträge
zusammenzuführen.

Das Stellen von Anträgen im Wiki ist dem Stellen von Anträgen im Pad
vorzuziehen, da nur hier eine Kontrolle des syntaktisch Aufbaus des
Wiki-Beschlussbausteins erfolgen kann. Im Pad gestellte Anträge wiesen
daher in der Vergangenheit öfters Syntaxfehler auf z.B. beim Protokoll
vom 2012-12-11, zudem enthalten die Pads praktisch immer halbe
(disfunktionale) Wiki-Beschlussbausteine.

Unstrukturierte Freitextanträge, wie sie im Pad möglich sind, oder
Anträge mit Syntaxfehlern, erhöhen den Aufwand ein brauchbares
Protokoll zu erstellen unnötig.

Zusatz Peda: Nur im Wiki kann sauber die Historie des Antrags
festgehalten werden und nachträgliche Ergänzungen, Änderungen aber
auch Vandalismus festgestellt werden.

Zusatz Tobi: keine doppelten Anträge bedingt durch mehrere Orte für
die Antragsstellung
\end{Begründung}

\Abstimmung{Christian "Bim" Reidel}{\ja}
\Abstimmung{Heide-Marie Weiherer}{\ja}
\Abstimmung{Walter Weber}{\neutral}
\Abstimmung{Otto Pittner}{\nichtda}

\end{Antrag}



\begin{Antrag}
        {Struktur der Entscheidungen}
        {\zurückgezogen}

\Antragssteller{3w\_SR}

\begin{Text}
Der Vorstand möge beschließen, in den Vorstandstelkos nur Anträge zu
beschließen, welche fristgemäß 24h vorher eingereicht wurden. Eine
Änderung des Antrags im Schrift und Wortlaut ist während der
Vorstandstelko nicht möglich.  Dies bedeutet, dass Anträge in dem
Schrift und Wortlaut und in dem Verständnis des Einzelnen entschieden
werden in dem diese fristgerecht vorliegen. Sollte der Steller des
Antrags mit dem Ergebnis unzufrieden sein, ist eine Diskussion
möglich, muß aber durch die einfache Mehrheit jeweils beschlossen
werden. Wird ein abgelehnter Antrag erneut in geänderter Form
eingereicht, sollte eine Versionsänderung vom Antragssteller
dokumentiert und nachvollziehbar zur Verfügung gestellt werden.
\end{Text}

\begin{Begründung}
Es ist für eine Bewertung und Umsetzung der entschiedenen Anträge
besser, in diesem Modus zu arbeiten, die Qualität der eingereichten
Anträge wird zunehmen.
\end{Begründung}

\Abstimmung{Christian "Bim" Reidel}{\neutral}
\Abstimmung{Heide-Marie Weiherer}{\neutral}
\Abstimmung{Walter Weber}{\neutral}
\Abstimmung{Otto Pittner}{\nichtda}

\end{Antrag}



\begin{Antrag}
        {Infostände für Landshut}
        {\abgelehnt}

\Antragssteller{Ahoi}

\begin{Text}
Ausstattung Stammtisch Landshut mit 1x Piratenflagge 2x3m (24,99), 1x
Piratenflagge 1x1,50m (9,50), 1x PVC-Banner (12,50), 1x
Tisch-Beach-Flag (28,50), 1x Grundsatzprogramm - Zeitung (500 St)
(103,35)
\end{Text}

\begin{Begründung}
Es ist armselig, an einem Infostand persönlichen Kontakt mit 100+
Leuten zu haben, dabei nicht eindeutig als Piraten erkannt zu werden
und Interessenten nichts mitgeben zu können.

Da der nächste Infostand ansteht, müsste die Bestellung sofort
erfolgen, damit das noch rechtzeitig ankommt.

Die Tisch-Beach-Flag wäre außerdem ein zusätzliches Erkennungsmerkmal
auf dem Tisch am Infostand und auch an jedem Stammtisch.
\end{Begründung}

\Abstimmung{Christian "Bim" Reidel}{\nein}
\Abstimmung{Heide-Marie Weiherer}{\nein}
\Abstimmung{Walter Weber}{\nein}
\Abstimmung{Otto Pittner}{\nichtda}

\end{Antrag}



\begin{Antrag}
        {Geschäftsstelle für Niederbayern}
        {\abgelehnt}

\Antragssteller{Ahoi}
\Unterstützer{F6F}

\begin{Text}
Der Vorstand soll die Geschäftsstelle nicht ohne eine
basisdemokratische Entscheidung der Piraten in Niederbayern nach
Offenlegung aller Informationen annehmen.
\end{Text}

\begin{Begründung}
Die Situation ist nicht klar und das Finanzielle nicht geregelt. Bevor
hier weiterer Aufwand getrieben wird, sollten erst alle
Rahmenbedingungen nachvollziehbar offengelegt werden.

Da auch die Nebenkosten ganz erheblich sei können, sollte das nicht
ohne die Basis entschieden werden.
\end{Begründung}


\Abstimmung{Christian "Bim" Reidel}{\nein}
    Einbindung der Basis ist selbstverständlich. Durch meine Ausführungen oben hat sich Antrag erledigt
\Abstimmung{Heide-Marie Weiherer}{\nein}
    der Antrag ist verfrüht. Falls sich eine Konkretisierung ergibt, müssen wir darüber sowieso ganz ausführlich sprechen.
\Abstimmung{Walter Weber}{\nein}
\Abstimmung{Otto Pittner}{\nichtda}

\end{Antrag}


\subsection{Sonstiges}
\begin{itemize}
\item Vorstellung des Konzeptes "Kommunalpiraten" von Bruno +
Planung/Verteilung der Aufgaben der Veranstaltung in
Niederbayern. Heiko wird deshalb auf der ML anfragen, welcher
Stammtisch das übernimmt.
\item Kurzvorstellung der zwei neuen Pads, die ich im Rahmen meiner
Beauftragung angelegt habe (Heiko):
  \begin{itemize}
  \item \url{http://piratenpad.de/p/presse-niederbayern}
  \item \url{http://piratenpad.de/p/wahlkampf-niederbayern}
  \item Heiko bittet alle Piraten, diese Pads an den jeweiligen
  Stammtischen zu kommunizieren, um sie möglichst mit Inhalt und
  Sachthemen zu füllen
  \end{itemize}

\item Wikischulung wird von F6F und Simon vorbereitet und abgehalten

\item Spenden / Verzicht auf Erstattung von Aufwendungen am Beispiel Pavilion
  \begin{itemize}
  \item Walter prüft, wie und ob das umzusetzbar ist
  \end{itemize}

\item Dringende Bitte an alle Stammtische Pressekontakte an Bim zu mailen

\item Walter wird auf der ML Niederbayern informieren
\end{itemize}

\subsection{Tagesordnungspunkte}

\textbf{Nächstes Treffen}
  \begin{itemize}
  \item Datum: 12.02.2013
  \item Uhrzeit: 19:30 Uhr 
  \item Ort: Mumble Raum Niederbayern
  \end{itemize}

\end{document}
